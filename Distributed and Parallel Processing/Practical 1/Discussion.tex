\documentclass[12pt,a4paper,oneside]{article}

\setlength{\textheight}{22cm}
\setlength{\textwidth}{16cm}
\setlength{\oddsidemargin}{0cm}
\setlength{\evensidemargin}{0cm}
\setlength{\topmargin}{0cm}
\setlength{\parindent}{0cm}
\setlength{\parskip}{0.8ex}
\usepackage{listings}
\usepackage{natbib}
\usepackage{fancyhdr}
\usepackage[english]{babel}
\usepackage[utf8]{inputenc}
\usepackage[normalem]{ulem}
\usepackage{setspace}
\usepackage{url}
\usepackage{graphicx}
\usepackage{float}
\usepackage{titlesec}
\useunder{\uline}{\ul}{}
\def\code#1{\texttt{#1}}
\setcounter{secnumdepth}{4}
\renewcommand\thesection{\arabic{section}}
%\AtBeginDocument{\renewcommand{\bibname}{References}}
\begin{document}

  \title{
    \\Distributed and Parallel Processing\\
    Practical 1 Writeup
  }
  \author{
    D.G. Smith - 13S0714
  }
  \date {1 August 2016}
  \pagenumbering{gobble}
  \pagenumbering{arabic}
  \maketitle
  \section{Mandelbrot}
    The code changes are uppercase comments in the MandelbrotThr.java source file.
    \\\\
    I used \code{ExecutorService} with a fixed thread count of five then used a
    \code{Callable} object to get the results obtained from the worker threads.
    A \code{Future} object is then used to get the actual results. I made use of this mechanism since the results of \code{calculateMandelbrot} needed to be obtained and passed to \code{display}. This ensures that the worker thread does not call \code{display} directly but is instead executed
    when the image starts being generated.
    \\\\
    The static class \code{WorkerThread} implements \code{Callable} instead
    of \code{Runnable}. This means that \code{compute} is implemented which allows
    a result to be returned (\code{run} does not allow a value to be returned).
    This simplified the returning of the Mandelbrot result each time it is calculated.
  \section{QuickSort}
    A parallel version of quicksort was implemented by inheriting the given class from
    \code{RecursiveAction} whilst keeping it generic. A threshold variable called
    \code{WORKLOAD} is used to determine whether new threads should be spawned
    using \code{fork} and \code{join} or whether to just simply keep it executing sequentially.
    \\\\
    In a test program, a \code{Integer} array is filled with 1 000 000
    elements with values between zero and one thousand. The mean execution time
    for a sequential sort is 782 ms. The mean parallel sort is 259 ms.
    This gives a speed up of approximately 66.88\%
\end{document}
